\documentclass{article}

\usepackage[round]{natbib}

\newcommand{\interp}[1]{[\![ #1 ]\!]}

\title{Towards Confluence: Model Building for First Order Abelian
  Logic using Backpropagation and Stochastic Gradient Descent}

\author{Daoud Clarke and Bill Keller}

\begin{document}

\maketitle

\section{Introduction}

In this paper we apply techniques used for learning neural networks to
build logical models. Specifically, we use backpropagation and
stochastic gradient descent to learn models for a new logic, first
order Abelian logic, which we also describe. The motivation for this
work is natural language semantics in which words are often
represented as vectors. The models we learn represent predicates as
vectors, or more generally, tensors.

\section{First order Abelian logic}

First order Abelian logic is intended to extend the predicate logic of
\cite{Meyer:89} with first order quantifiers, however we leave the
proof of a formal relationship between their logic and ours to further
work. Moreover, we use vector spaces for models instead of the more
general Abelian groups they use, since the continuity properties are
more suited to the techniques we wish to apply.

The language of first order Abelian logic is that of standard first
order logic; for simplicity we exclude equality, functions and free
variables. We denote variables $x_1, x_2, \ldots$. Its
semantics are as follows:
\begin{itemize}
\item The interpretation of an $n$ place predicate symbol $P$ is a
  tensor $T_P$ of order $n$; zero-place predicates (constants) are
  zero order tensors (scalars).
\item The interpretation of a formula is an \emph{assigned tensor},
  i.e.~a pair $\langle T, A\rangle$, where $T$ is a tensor of order
  $n$ and $A$ is an $n$-tuple of integers describing variable
  assignment.
\item The interpretation of a formula $P(x_{i_1}, x_{i_2},\ldots
  x_{i_n})$ where $P$ is an $n$-place predicate symbol is the assigned
  tensor $\langle T_P, \langle i_1, i_2, \ldots i_n\rangle\rangle$. For
  example the formula $\mathit{loves}(x_2, x_3)$ would have the
  assigned tensor $\langle T_{\mathit{loves}}, \langle 2, 3
  \rangle\rangle$.
\item The interpretation of $\alpha \circ \beta$, where $\circ$ is
  either $\land$, $\lor$ or $\rightarrow$ is
  $\interp{\alpha}\bullet\interp{\beta}$, where $\bullet$ is the
  \emph{assigned tensor unification} operation corresponding to
  $\circ$, described below, and $\interp{\alpha}$ denotes the
  interpretation of the formula $\alpha$.
\item The interpretation of $\forall x_i\ \alpha$ (or $\exists
  x_i\ \alpha$) where $\interp{\alpha} = \langle T, A\rangle$ and $T$
  is of order $n$ is an assigned tensor $\langle T', A'\rangle$ where
  $T'$ is of order $n-1$ and consists of the component-wise minimum
  (component-wise maximum) of all slices of the tensor in dimension
  $j$, where $j$ is the index of $A$ in which $i$ occurs and $A'$ is
  $A$ with $i$ removed. For example, suppose the predicate
  $\mathit{loves}$ is described by the order 2 tensor
$$T_{\mathit{loves}} = \left( \begin{array}{rrr}
0.1 & 0 & -0.2 \\
0.2 & 0.5 & 0 \\
-0.3 & 0.1 & 0 \end{array} \right)$$
Then $\interp{\mathit{loves}(x_3, x_2)} = \langle{T_{\mathit{loves}},
  \langle 3,2\rangle\rangle}$ and $\interp{\exists
  x_2\ \mathit{loves}(x_3, x_2)}$ is $\langle T_{\mathit{loves}}',
\langle 3\rangle\rangle$, where
$$T_{\mathit{loves}}' = \left( \begin{array}{r}
0.1 \\
0.2 \\
-0.3
\end{array}\right) \sqcup \left( \begin{array}{r}
0 \\
0.5 \\
0.1 
\end{array}\right) \sqcup \left( \begin{array}{r}
-0.2 \\
0 \\
0 \end{array} \right) = \left( \begin{array}{r}
0.1 \\
0.5 \\
0.1 \end{array} \right)$$
and $\sqcup$ denotes the component-wise maximum. Note that we perform
the operation over the \emph{columns} of the tensor since the
quantification is over $x_2$ and 2 occurs in the second position of
$A$; if the quantification had been over $x_3$ we would have used the
rows.

\end{itemize}

%% \item The interpretation of an $n$ place predicate applied to
%%   variables $x_{i_1}, x_{i_2}, \ldots x_{i_n}$ is a transposition of the
%%   tensor such that transposing the corresponding variables results in
%%   the variables being ordered from least to greatest.
%% \item $\interp{\alpha \land \beta} = \interp{\alpha} \land
%%   \interp{\beta}$ where $\land$ on the right hand side denotes
%%   the tensor component-wise minimum.
%% \item $\interp{\alpha \lor \beta} = \interp{\alpha} \lor
%%   \interp{\beta}$ where $\lor$ on the right hand side denotes
%%   the tensor component-wise maximum.

%% where, for $T$ a tensor of order $n$, $\bigvee_i T$ denotes the tensor
%% of order $n-1$ formed by taking the component-wise maximum over all
%% order $n-1$ sub-tensors in the $i$th dimension. Similarly,
%% $\bigwedge_i$ T denotes the tensor formed by taking component-wise
%% minimums over all sub-tensors in the $i$th dimension.


\subsection{Assigned tensor unification}

We define the operations $\sqcup$, $\sqcap$ and $-$ between assigned
tensors, these are used in interpretations in place of the logical
symbols $\lor$, $\land$ and $\leftarrow$ respectively. Let $\langle
T, A\rangle$ and $\langle S, B\rangle$ be two assigned tensors. We
define $\langle U, C\rangle = \langle T, A\rangle \bullet \langle S,
B\rangle$ where $\bullet$ is one of $\sqcup$, $\sqcap$ and $-$. $C$ is
the sorted union of $A$ and $B$ in ascending order. Let $I = \langle
i_1, i_2 \ldots i_m\rangle$ be an indexing of $U$. The components
 of $U$ are
$$ U_I = T_J \bullet S_K$$
where $J$ and $K$ are indexes defined such that $j_a = i_c$ and $k_b =
i_c$ where $c$ is the index in $C$ of the $a$th element of $A$ and $b$
is its index in $B$.



\bibliographystyle{plainnat}
\bibliography{JW2012}


\end{document}
